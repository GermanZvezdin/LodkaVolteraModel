\documentclass[a4paper,12pt,titlepage,finall]{article}

\usepackage[T1,T2A]{fontenc}     % форматы шрифтов
\usepackage[utf8x]{inputenc}     % кодировка символов, используемая в данном файле
\usepackage[russian]{babel}      % пакет русификации
\usepackage{tikz}                % для создания иллюстраций
\usepackage{pgfplots}            % для вывода графиков функций
\usepackage{geometry}		 % для настройки размера полей
\usepackage{indentfirst}         % для отступа в первом абзаце секции
\usepackage{listings}
\usepackage{amsmath}
\usepackage{amssymb}
\usepackage{graphicx}
\graphicspath{ {./images/} }
\usepackage{xcolor}

\definecolor{codegreen}{rgb}{0,0.6,0}
\definecolor{codegray}{rgb}{0.5,0.5,0.5}
\definecolor{codepurple}{rgb}{0.58,0,0.82}
\definecolor{backcolour}{rgb}{0.95,0.95,0.92}

\lstdefinestyle{mystyle}{
    backgroundcolor=\color{backcolour},   
    commentstyle=\color{codegreen},
    keywordstyle=\color{magenta},
    numberstyle=\tiny\color{codegray},
    stringstyle=\color{codepurple},
    basicstyle=\ttfamily\footnotesize,
    breakatwhitespace=false,         
    breaklines=true,                 
    captionpos=b,                    
    keepspaces=true,                 
    numbers=left,                    
    numbersep=5pt,                  
    showspaces=false,                
    showstringspaces=false,
    showtabs=false,                  
    tabsize=2
}
\lstset{style=mystyle}
% выбираем размер листа А4, все поля ставим по 3см
\geometry{a4paper,left=30mm,top=30mm,bottom=30mm,right=30mm}

\setcounter{secnumdepth}{0}      % отключаем нумерацию секций

\usepgfplotslibrary{fillbetween} % для изображения областей на графиках

\begin{document}
% Титульный лист
\begin{titlepage}
    \begin{center}
	{\small \sc Московский государственный университет \\имени М.~В.~Ломоносова\\
	Факультет вычислительной математики и кибернетики\\}
	\vfill
	{\Large \sc Отчет по заданию №1}\\
	~\\
	{\large \bf <<Численное решение модельной задачи Лодка-Вольтерра.\\
	    Метод Рунге — Кутты>>}\\ 
	~\\
    \end{center}
    \begin{flushright}
	\vfill {Выполнил:\\
	студент 305 группы\\
	Звездин~Г.~Е.\\
	~\\
	Преподаватель:\\
	Есикова~Н.~Б.}
    \end{flushright}
    \begin{center}
	\vfill
	{\small Москва\ 2021}
    \end{center}
\end{titlepage}

% Автоматически генерируем оглавление на отдельной странице
\tableofcontents
\newpage

\section{Постановка задачи}

\begin{itemize}
\item Задача заключается в нахождении решения системы обыкновенных дифференциальных уравнений первого порядка методом Рунге — Кутты.

\begin{equation*}
\begin{cases}
$$\dot{y_1}$ = y_1(E_1-y_2)$ & \text{E_i > 0, $\mu$ > 0, $y_i \geqslant 0$,  $0 \leqslant t \leqslant 100$,  $i = \overline{1..3}$}
\\
$$\dot{y_2}$ = y_2(-E_2+y_1-y_3)$ 
\\
$$\dot{y_3}$ = y_3(-E_3+\mu y_2)$
\end{cases}
\end{equation*} 

\item Нахождение стационарных точек. \newline
$\dot{y_i} = 0$, $i = \overline{1..3}$ \newline
Получим: 
\begin{equation*}
\begin{cases}
$0 = y_1(E_1-y_2)$ & \text{E_i > 0, $\mu$ > 0, $y_i \geqslant 0$,  $0 \leqslant t \leqslant 100$,  $i = \overline{1..3}$}
\\
$0 = y_2(-E_2+y_1-y_3)$ 
\\
$0 = y_3(-E_3+\mu y_2)$

\end{cases}
\end{equation*} \newline
Из системы получаем тривиальное решение: \newline
$y_1 = y_2 = y_3 = 0$ \newline
Так же находим беконечное множество решений: \newline
$y_2 = E_1$, $y_1 = E_2 + y_3$, $y_3 \in \mathbb{R}$  \newline
Так же заметим, что из уравнения 3 следует, что $E_3 = \mu E_1$ и нарушение этого условия приведет к отсутствию нетривиального решения. 

\item Реализация численного метода для решения данной задачи. 
\item Изучение динамики системы с начальными условиями: $y_i(0)= y^* + \epsilon$, где $0 < \epsilon < 1$, с различными $E_i, \mu, i = \overline{1..3}$
\item Нахождение переодического решения с периодом $\lambda \approx 11$
\newpage

\section{Реализация численного метода для решения данной задачи}
Будем использовать метод Рунге-Кутты 4-го порядка точности: \newline
$y_i_k = y_{i-1}_k + \frac{h}{6}(k_1 + 2k_2 + 2k_3 + k_4)$, где $k_1, k_2, k_3, k_4$ \newline 
вычисляются следующим образом: \newline
$k_1 = f_i(t, y_1, y_2, y_3)$,   $i = \overline{1..3}$ \newline
$k_2 = f_i(t +\frac{h}{2}, y_1 + k_1 \frac{h}{2}, y_2 + k_1 \frac{h}{2}, y_3 + k_1 \frac{h}{2})$ \newline
$k_3 =f_i(t +\frac{h}{2}, y_1 + k_2\frac{h}{2}, y_2 + k_2 \frac{h}{2}, y_3 + k_2 \frac{h}{2})$ \newline
$k_4 =f_i(t +h, y_1 + k_3 h, y_2 + k_3 h, y_3 + k_3 h)$ \newline
Последовательно вычисляются приближающие угловые коэффициенты: $k_1$ – в исходной точке, $k_2$ – на половинном шаге, $k_3$ – тоже на половинном шаге, но по уточнённому значению углового коэффициента $k_2$ вместо $k_1$, $k_4$ – на целом шаге по предыдущему значению $k_3$. Стоит отметить, что получаемые здесь на каждом этапе $k_1, k_2, k_3, k_4$ – угловые коэффициенты четырёх разных интегральных кривых в трёх точках: $t_i, t_i + \frac{h}{2}, t_{i+1} = t_i + h$. Для получения
нового значения искомой функции на полном шаге используется взвешенное среднее этих
угловых коэффициентов. Так же заметим, что классический метод Рунге-Кутты является явным и допускает расчёт с переменным шагом.
\newpage
\section{Численное решение задачи}
Используя вышеописанный метод решим исходню систему для заданных параметров:
$E_1 = 0.6, E_2 = 0.58, E_3 = 0.12, \mu = 0.2, h = 0.01$ и начальных условиях: \\
$y_1(0) = 0 + \epsilon, y_2(0) = 0 + \epsilon, y_3(0)= 0 + \epsilon, \epsilon = 0.25$


\begin{figure}[h]
    \centering
    \includegraphics[scale=0.4]{images/Y1(T).png}
    \caption{$График y_1(t)$}
    \label{fig:my_label}
\end{figure} 


\begin{figure}[h]
    \centering
    \includegraphics[scale=0.4]{images/y2(t)_E1_0.6_E2_0.58_E3_0.12_MU_0.2_H_0.01.png}
    \caption{$График y_2(t)$}
    \label{fig:my_label}
\end{figure}

\newpage
\begin{figure}[h]
    \centering
    \includegraphics[scale=0.4]{images/y3(t)_E1_0.6_E2_0.58_E3_0.12_MU_0.2_H_0.01.png}
    \caption{$График y_3(t)$}
    \label{fig:my_label}
\end{figure}

\begin{figure}[h]
    \centering
    \includegraphics[scale=0.4]{images/y1(y2)_E1_0.6_E2_0.58_E3_0.12_MU_0.2_H_0.01.png}
    \caption{$График y_1(y_2)$}
    \label{fig:my_label}
\end{figure}

\newpage
Найдено переодическое решение с периодом $\lambda \approx 11$ и параметрах:
$E_1 = 0.1, E2 = 1.2, E_3 = 0.06, \mu = 0.6, h = 0.01$ и начальных условиях: \newline
$y_1(0) = 0.1 + \epsilon, y_2(0) = 0.2 + \epsilon, y_3(0)= 1 + \epsilon, \epsilon = 0.25$

\begin{figure}[h]
    \centering
    \includegraphics[scale=0.4]{images/y1(t)_E1_0.1_E2_1.2_E3_0.06_MU_0.6_H_0.01.png}
    \caption{$График y_1(t)$}
    \label{fig:my_label}
\end{figure}

\begin{figure}[h]
    \centering
    \includegraphics[scale=0.4]{images/y2(t)_E1_0.6_E2_0.75_E3_0.006_MU_0.01_H_0.01.png}
    \caption{$График y_2(t)$}
    \label{fig:my_label}
\end{figure}

\newpage

\begin{figure}[h]
    \centering
    \includegraphics[scale=0.4]{images/y3(t)_E1_0.6_E2_0.75_E3_0.006_MU_0.01_H_0.01.png}
    \caption{$График y_3(t)$}
    \label{fig:my_label}
\end{figure}


\begin{figure}[h]
    \centering
    \includegraphics[scale=0.4]{images/y1(y2)_E1_0.6_E2_0.75_E3_0.006_MU_0.01_H_0.01.png}
    \caption{$График y_1(y_2)$}
    \label{fig:my_label}
\end{figure}

\newpage

\begin{figure}[h]
    \centering
    \includegraphics[scale=0.4]{images/y1(y3)_E1_0.6_E2_0.75_E3_0.006_MU_0.01_H_0.01.png}
    \caption{$График y_1(y_3)$}
    \label{fig:my_label}
\end{figure}


\begin{figure}[h]
    \centering
    \includegraphics[scale=0.4]{images/y2(y3)_E1_0.6_E2_0.75_E3_0.006_MU_0.01_H_0.01.png}
    \caption{$График y_2(y_3)$}
    \label{fig:my_label}
\end{figure}
\newpage

\newpage
\section{Исходный код}
\begin{lstlisting}[language=python, caption={Исходный код программы}]

#
# Created by German Zvezdin on 13.02.2021.
#


E1 = 0.1
E2 = 1.2
MU = 0.6
E3 = MU * E1
h = 0.01

def f1(y1, y2):
    return y1 * (E1 - y2)
def f2(y1, y2, y3):
    return y2 * (-E2 + y1 - y3)
def f3(y2, y3):
    return y3 * (-E3 + MU * y2)


def k1(func, x, y):
    return func(x, y)
def k1m(func, x, y, z):
    return func(x, y, z)

def k2(func,  x,  y,  k):
    return func(x + h/2, y + k*h/2)
def k2m(func,  x,  y,  z,  k):
    return h * func(x + h/2, y + k/2, z + k*h/2)

def k3(func, x, y, k2):
    return func(x + h / 2, y + k2 * h / 2)
def k3m(func,  x,  y, z,  k2):
    return func(x + h / 2, y + k2 * h / 2, z + k2 * h / 2)


def k4(func,  x,  y, k3):
    return func(x + h, y + h * k3)
def k4m(func,  x,  y,  z, k3):
    return func(x + h, y + h * k3, z + h * k3)


def NexStepValue( y_prev,  k1,  k2,  k3,  k4):
    return y_prev + h * (k1 + 2 * k2 + 2 * k3 + k4) / 6

if __name__ == "__main__" :

    y_1 = E2 - 1 + 0.25
    y_2 = E1 + 0.25
    y_3 = 1.0 + 0.25
    t = 0
    
    while(t <= 100.0):

        ck1 = k1(f1, y_1, y_2)
        ck2 = k2(f1, y_1, y_2, ck1)
        ck3 = k3(f1, y_1, y_2, ck2)
        ck4 = k4(f1, y_1, y_2, ck3)


        forY1 = [ck1, ck2, ck3, ck4]

        ck1 = k1m(f2, y_1, y_2, y_3)
        ck2 = k2m(f2, y_1, y_2, y_3, ck1)
        ck3 = k3m(f2, y_1, y_2, y_3, ck2)
        ck4 = k4m(f2, y_1, y_2, y_3, ck3)


        forY2 = [ck1, ck2, ck3, ck4]

        ck1 = k1(f3, y_2, y_3)
        ck2 = k2(f3, y_2, y_3, ck1)
        ck3 = k3(f3, y_2, y_3, ck2)
        ck4 = k4(f3, y_2, y_3, ck3)


        forY3 = [ck1, ck2, ck3, ck4]

        y_1 = NexStepValue(y_1, forY1[0], forY1[1], forY1[2], forY1[3])

        y_2 = NexStepValue(y_2, forY2[0], forY2[1], forY2[2], forY2[3])

        y_3 = NexStepValue(y_3, forY3[0], forY3[1], forY3[2], forY3[3])

        t += h


\end{lstlisting}
\end{document}